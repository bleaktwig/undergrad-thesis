Since the project consists on the heavy optimization of the DC code, a logical starting point is a detailed profiling of the compiled program in order to figure out where the performance bottlenecks are found so as to focus work on them. % Note: This could use a citation.
Different profiling programs or \textbf{profilers} were evaluated for this purpose, but in the end the default java profiler, or \textbf{jvisualvm}, was used.
A small analysis into the reasoning behind this decision is shown in Section \ref{ssec:prof_tools} and then the profiling results and a short analysis of them is given in Section \ref{ssec:prof_results}.

% DEFINE PROFILING.
Profiling or ``software profiling'' is a dynamic program analysis method that measures all factors to analyze the performance of an application.
These factors were separated into four categories: CPU/GPU usage, memory usage, IO and network operations and database queries.
These categories are used because they encompass all the processes that can be commonly found in an application.

A \textbf{profiler} is a tool that analyzes either the source code or the running program to report its performance after or while it is running, usually offering information like the program's total memory use, the number of times a certain method was called or the largest bottlenecks in the programs execution~\cite{ball1997efficient}.
% EXPLAIN PROFILING VIA SAMPLING VS INSTRUMENTING.
There are three ways to obtain profiling information: \textbf{code instrumentation}, \textbf{statistical sampling} and \textbf{application monitoring}.

\textbf{Code instrumentation} is the act of placing new lines of code in a program and counts how many times they were ran, so as to give the exact number while the program is running or after it is completed~\cite{ball1994optimally}.
Instrumentation can give a very detailed view into exactly how the program is working, but due to the fact that new lines of code are added to the program, the execution time can drastically increase while profiling is taking place.
An important terminological consideration is that code instrumentation is usually simply referred to as profiling, which can lead to confusion when studying the area. % Note: This could use a citation but it comes from reading a lot about the subject online.

In stark contrast, \textbf{statistical sampling} is a less disruptive method that work by stopping the program in regular intervals during its execution, taking a so-called ``snapshot'' and then allowing the application to continue, sacrificing accuracy in favor of speed~\cite{wenisch2006simflex}.
Summarized, sampling may reveal the relative percentage of time spent in frequently-called methods, but cannot provide the exact number of times each method is invoked, unlike instrumentation.

\textbf{Application monitoring} provides a so-called ``eagle's point of view'' onto a running application, focusing on the high-level picture of it.
Monitoring provides very basic information about an application but is the least disruptive of all the methods mentioned, and thus is mostly used for web services that require reports onto how they're running while also minimizing the disruption this analysis has on the running program so as to not impact its performance~\cite{hunt2011java}.

As mentioned before, all of the effort done in this work is based on multiple instances of performance analysis done via the \textbf{jvisualvm} tool.
Also, all the results given in Section \ref{ssec:prof_results} and chapter $6$ are obtained by measuring the total execution time of the program with the tool in a personal computer, an Intel(R) Core(TM) i$5$-$6600$K CPU with $4$ cores running at $3.50$GHz with $8$GB of DDR$4$ $2133$MHz RAM. % CHAPTER REFERENCE