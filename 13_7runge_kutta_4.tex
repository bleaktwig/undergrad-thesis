\subsection{Fourth Order Runge Kutta} \label{ssec:framework_rk4} 
The Fourth Order Runge Kutta method or simply Runge Kutta 4 (RK4) is an iterative method used to solve initial value problems, originally proposed by Carl Runge and Wilhelm Kutta~\cite{runge1895numerische}.
In CLAS12, the method is used to estimate a state vector $\mathbf{x}(z+h)$ between two measurement planes $z_{k-1}$ and $z_k$, denoted as $z_{k-1} + h$, where $h$ is the step size used, given that $z_k > z_{k-1} + h$, and thus $\mathbf{x}(z+h)$ is between $\mathbf{x}_{k-1|k-1}$ and $\mathbf{x}_{k|k-1}$.

Let an initial value problem for estimating $\mathbf{x}(z+h)$ be:
    \begin{equation*}
        \mathbf{\dot{x}} = \mathbf{f}(z,\mathbf{x}), \mathbf{x}(z_0) = \mathbf{x}_0\,,
    \end{equation*}
where $\mathbf{x}$ is the unknown vector function of the measurement plane $z$ to be approximated, $\mathbf{\dot{x}}$ is its rate of change, defined as a function of $z$ and $\mathbf{x}$ itself.
$\mathbf{f}$ denotes the equations of motion from Section \ref{ssec:framework_tf} and the initial $z_0$ and $\mathbf{x}_0$ are given.
Using the step size $h > 0$ defined before:
    \begin{equation}
        \mathbf{x}(z+h) = \mathbf{x}(z) + \frac{1}{6}(\mathbf{r}_1 + \mathbf{r}_2 + \mathbf{r}_3 + \mathbf{r}_4)\,,\label{eq:framework_rk4_x}
    \end{equation}
with $\mathbf{r}_1$, $\mathbf{r}_2$, $\mathbf{r}_3$ and $\mathbf{r}_4$ defined as:
    \begin{align*}
        \mathbf{r}_1 &= h\cdot\mathbf{f}(z, \mathbf{x}(z))\,,\\
        \mathbf{r}_2 &= h\cdot\mathbf{f}\left(z + \frac{h}{2}, \mathbf{x}(z) + \frac{\mathbf{r}_1}{2}\right),\\
        \mathbf{r}_3 &= h\cdot\mathbf{f}\left(z + \frac{h}{2}, \mathbf{x}(z) + \frac{\mathbf{r}_2}{2}\right),\\
        \mathbf{r}_4 &= h\cdot\mathbf{f}(z + h, \mathbf{x}(z) + \mathbf{r}_3)\,.
    \end{align*}
Finally, the $\mathbf{x}(z+h)$ presented in Equation \eqref{eq:framework_rk4_x} is named the RK4 approximation of $\mathbf{x}(z+h)$~\cite{sauer2012numerical}.

The pseudocode for the implementation of RK4 at the DC software is presented:

\newpage

    \begin{algorithm}[H]
        \caption{Runge Kutta 4}
        \SetAlgoLined
        \DontPrintSemicolon
        \SetKwProg{Def}{def}{:}{}
        $z \gets$ initial measurement location\;
        $h \gets$ step size\;
        $\mathbf{s}_k \gets$ state vector at $k$\;
        $S_k \gets$ covariance matrix at $k$\;
        $\chi^2 \gets \chi^2$ error\;
        \Def{\upshape rk4\_transport$(z, h, \mathbf{s}_k, S_k, \chi^2)$}{
            \CommentSty{// $r_1$}\;
            $\mathbf{B} \gets$ bfield at location\;
            $\mathbf{k}_{\mathbf{x}(1)} \gets \mathbf{f}_{\mathbf{x}}(z, \mathbf{x}, \mathbf{B})$\;
            $K_{J(1)} \gets F_J(z, \mathbf{x}, J, \mathbf{B})$\;
            \CommentSty{// $r_2$}\;
            $\mathbf{B} \gets$ bfield at new location\;
            $\mathbf{k}_{\mathbf{x}(2)} \gets \mathbf{f}_{\mathbf{x}}(z+\frac{h}{2}, \mathbf{x}+\frac{h}{2}\mathbf{k}_{\mathbf{x}(1)}, \mathbf{B})$\;
            $K_{J(2)} \gets F_J(z+\frac{h}{2}, \mathbf{x}+\frac{h}{2}\mathbf{k}_{\mathbf{x}(1)}, J+\frac{h}{2}K_{J(1)}, \mathbf{B})$\;
            \CommentSty{// $r_3$}\;
            $\mathbf{B} \gets$ bfield at new location\;
            $\mathbf{k}_{\mathbf{x}(3)} \gets \mathbf{f}_{\mathbf{x}}(z+\frac{h}{2}, \mathbf{x}+\frac{h}{2}\mathbf{k}_{\mathbf{x}(2)}, \mathbf{B})$\;
            $K_{J(3)} \gets F_J(z+\frac{h}{2}, \mathbf{x}+\frac{h}{2}\mathbf{k}_{\mathbf{x}(2)}, J+\frac{h}{2}K_{J(2)}, \mathbf{B})$\;
            \CommentSty{// $r_4$}\;
            $\mathbf{B} \gets$ bfield at new location\;
            $\mathbf{k}_{\mathbf{x}(4)} \gets \mathbf{f}_{\mathbf{x}}(z+h, \mathbf{x}+h\mathbf{k}_{\mathbf{x}(3)}, \mathbf{B})$\;
            $K_{J(4)} \gets F_J(z+h, \mathbf{x}+h\mathbf{k}_{\mathbf{x}(3)}, J+h K_{J(3)}, \mathbf{B})$\;
            \CommentSty{// RK4}\;
            $z \gets z + h$\;
            $\mathbf{x} \gets \mathbf{x} + \frac{h}{6} (\mathbf{k}_{\mathbf{x}(1)} + 2\mathbf{k}_{\mathbf{x}(2)} + 2\mathbf{k}_{\mathbf{x}(3)} + \mathbf{k}_{\mathbf{x}(4)})$\;
            $J \gets J + \frac{h}{6} (K_{J(1)} + 2K_{J(2)} + 2K_{J(3)} + K_{J(4)})$\;
            $S_k \gets$ update\_cov\_mat$(S_k, J)$\;
            $S_k \gets$ q\_process\_estimate$(S_k, z, h, \mathbf{x}, \text{mass})$\;
            \Return{$z,\mathbf{x},S_k,\chi^2$}
        }
    \end{algorithm}
    
where $\mathbf{f}_{\mathbf{x}}(z, \mathbf{x}, \mathbf{B})$ and $F_J(z, \mathbf{x}, J, \mathbf{B})$ are the equations of motion described in Section \ref{ssec:framework_tf}, \texttt{update\_cov\_mat}($S_k, J$) applies the equations of motion of the Jacobian matrix to the covariance matrix and \texttt{q\_process\_estimate}($S_k, z, h, \mathbf{x}, \text{mass}$) updates it from the estimated noise in $q$.
More detail on this can be seen in the original source code from the repository linked in Appendix \ref{add:code_and_reproducibility}.