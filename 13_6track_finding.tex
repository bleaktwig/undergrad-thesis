\subsection{Track Finding} \label{ssec:framework_tf}
In the Drift Chambers software, each particle's track is estimated independently via the EKF, process that is dubbed \textbf{Track Fitting}.
The state vector estimated by the EKF is $\mathbf{x}_k(z)$, and is defined as:
    \begin{equation*}
        \mathbf{x}_k(z) = \begin{pmatrix}x\\ y\\ \theta_x\\ \theta_y\\ q\end{pmatrix}\,,
    \end{equation*}
given that:
    \begin{equation*}
        \theta_x = \frac{\text{p}_x}{\text{p}_z},
        \theta_y = \frac{\text{p}_y}{\text{p}_z},
        q = \frac{Q_e}{\|\mathbf{p}\|}\,,
    \end{equation*}
    
where ($x,y$) represents the coordinates of the particle measured in a specific plane $z$, commonly denominated \textbf{measurement plane}.
A measurement plane $z$ can be thought of as a plane perpendicular to the beam line seen in Figure \ref{fig:dc_horizontal_cut} producing a view similar to the one seen in Figure \ref{fig:dc_vertical_cut}.
Unlike traditional methods, $z$ is used as the ``time'' variable because the speed of each particle in $z$ is constant for this purpose.
$\mathbf{p}=(p_x,p_y,p_z)$ is the momentum of the particle in that measurement plane and $Q_e$ is the electric charge of the particle. The equations of motion
% , expressed as derivatives with respect to $z$, 
are:
    \begin{align*}
        f_x(z, \mathbf{x}) &= \theta_x\\
        f_y(z, \mathbf{x}) &= \theta_y\\
        \nonumber f_{\theta_x}(z, \mathbf{x}) &= q\, v \sqrt{1+\theta_x^2+\theta_y^2}\\
        &~~\left(\theta_y(\theta_x \operatorname{B_{1}}(x,y,z)+\operatorname{B_{3}}(x,y,z) - (1+\theta_x^2) \operatorname{B_{2}}(x,y,z)\right)\\
        \nonumber f_{\theta_y}(z, \mathbf{x}) &= q\, v \sqrt{1+\theta_x^2+\theta_y^2}\\ 
        &~~\left(-\theta_x (\theta_y\, \operatorname{B_{2}}(x,y,z)+\operatorname{B_{3}}(x,y,z) + (1+\theta_y^2) \operatorname{B_{1}}(x,y,z)\right)\\
        f_q(z, \mathbf{x}) &= 0\,,
    \end{align*}

where $\textbf{B}(x,y,z) = (\operatorname{B_{1}}(x,y,z), \operatorname{B_{2}}(x,y,z), \operatorname{B_{3}}(x,y,z))$ is the effect of the magnetic field in a particular $(x,y,z)$ position inside the detector and $v$ is the speed of light in the medium.
The components of $\textbf{B}$ are defined as $\operatorname{B_{1}}$, $\operatorname{B_{2}}$ and $\operatorname{B_{3}}$ instead of $\operatorname{B_{x}}$, $\operatorname{B_{y}}$ and $\operatorname{B_{z}}$ to avoid confusion with the notation of partial derivatives.
For the EKF to be able to be solved, the gradient of these equations of motion are also obtained:
    \begin{align*}
        \frac{\partial f_x}{\partial x} = 0,~
        \frac{\partial f_x}{\partial y} = 0,~
        \frac{\partial f_x}{\partial \theta_x} &= 1,~
        \frac{\partial f_x}{\partial \theta_y} = 0,~
        \frac{\partial f_x}{\partial q} = 0\\
        \frac{\partial f_y}{\partial x} = 0,~
        \frac{\partial f_y}{\partial y} = 0,~
        \frac{\partial f_y}{\partial \theta_x} &= 0,~
        \frac{\partial f_y}{\partial \theta_y} = 1,~
        \frac{\partial f_y}{\partial q} = 0\,,
    \end{align*}
    \begin{align}
        \frac{\partial f_{\theta_x}}{\partial x} &= 
        q\, v \sqrt{\theta_{x}^{2} + \theta_{y}^{2} + 1} \nonumber\\ 
        &~~\left(\theta_{y} \left(\theta_{x} \frac{\partial}{\partial x} \operatorname{B_{1}}{\left (x,y,z \right)} + \frac{\partial}{\partial x} \operatorname{B_{3}}{\left (x,y,z \right )}\right) + \left(- \theta_{x}^{2} - 1\right) \frac{\partial}{\partial x} \operatorname{B_{2}}{\left (x,y,z \right )}\right) \label{eq:tf_dtx_dx}\\
        \frac{\partial f_{\theta_x}}{\partial y} &=
        q\, v \sqrt{\theta_{x}^{2} + \theta_{y}^{2} + 1} \nonumber\\
        &~~\left(\theta_{y} \left(\theta_{x} \frac{\partial}{\partial y} \operatorname{B_{1}}{\left (x,y,z \right )} + \frac{\partial}{\partial y} \operatorname{B_{3}}{\left (x,y,z \right )}\right) + \left(- \theta_{x}^{2} - 1\right) \frac{\partial}{\partial y} \operatorname{B_{2}}{\left (x,y,z \right )}\right) \label{eq:tf_dtx_dy}\\
        \frac{\partial f_{\theta_x}}{\partial \theta_x} &=
        \frac{q\, \theta_{x}\, v \left(\theta_{y} \left(\theta_{x} \operatorname{B_{1}}{\left (x,y,z \right )} + \operatorname{B_{3}}{\left (x,y,z \right )}\right) - \left(\theta_{x}^{2} + 1\right) \operatorname{B_{2}}{\left (x,y,z \right )}\right)}{\sqrt{\theta_{x}^{2} + \theta_{y}^{2} + 1}} \nonumber\\
        &~~ + q\, v \left(- 2 \theta_{x} \operatorname{B_{2}}{\left (x,y,z \right )} + \theta_{y} \operatorname{B_{1}}{\left (x,y,z \right )}\right) \sqrt{\theta_{x}^{2} + \theta_{y}^{2} + 1} \nonumber\\
        \frac{\partial f_{\theta_x}}{\partial \theta_y} &=
        \frac{q\, \theta_{y}\, v \left(\theta_{y} \left(\theta_{x} \operatorname{B_{1}}{\left (x,y,z \right )} + \operatorname{B_{3}}{\left (x,y,z \right )}\right) - \left(\theta_{x}^{2} + 1\right) \operatorname{B_{2}}{\left (x,y,z \right )}\right)}{\sqrt{\theta_{x}^{2} + \theta_{y}^{2} + 1}} \nonumber\\
        &~~ + q\, v \left(\theta_{x} \operatorname{B_{1}}{\left (x,y,z \right )} + \operatorname{B_{3}}{\left (x,y,z \right )}\right) \sqrt{\theta_{x}^{2} + \theta_{y}^{2} + 1} \nonumber\\
        \frac{\partial f_{\theta_x}}{\partial q} &=
        v \left(\theta_{y} \left(\theta_{x} \operatorname{B_{1}}{\left (x,y,z \right )} + \operatorname{B_{3}}{\left (x,y,z \right )}\right) - \left(\theta_{x}^{2} + 1\right) \operatorname{B_{2}}{\left (x,y,z \right )}\right) \sqrt{\theta_{x}^{2} + \theta_{y}^{2} + 1} \nonumber
    \end{align}
    \begin{align}
        \frac{\partial f_{\theta_y}}{\partial x} &=
        q\,v \sqrt{\theta_{x}^{2} + \theta_{y}^{2} + 1} \nonumber\\
        &~~ \left(- \theta_{x} \left(\theta_{y} \frac{\partial}{\partial x} \operatorname{B_{2}}{\left (x,y,z \right )} + \frac{\partial}{\partial x} \operatorname{B_{3}}{\left (x,y,z \right )}\right) + \left(\theta_{y}^{2} + 1\right) \frac{\partial}{\partial x} \operatorname{B_{1}}{\left (x,y,z \right )}\right) \label{eq:tf_dty_dx}\\
        \frac{\partial f_{\theta_y}}{\partial y} &=
        q\, v \sqrt{\theta_{x}^{2} + \theta_{y}^{2} + 1} \nonumber\\
        &~~ \left(- \theta_{x} \left(\theta_{y} \frac{\partial}{\partial y} \operatorname{B_{2}}{\left (x,y,z \right )} + \frac{\partial}{\partial y} \operatorname{B_{3}}{\left (x,y,z \right )}\right) + \left(\theta_{y}^{2} + 1\right) \frac{\partial}{\partial y} \operatorname{B_{1}}{\left (x,y,z \right )}\right) \label{eq:tf_dty_dy}\\
        \frac{\partial f_{\theta_y}}{\partial \theta_x} &=
        \frac{q\, \theta_{x}\, v \left(- \theta_{x} \left(\theta_{y} \operatorname{B_{2}}{\left (x,y,z \right )} + \operatorname{B_{3}}{\left (x,y,z \right )}\right) + \left(\theta_{y}^{2} + 1\right) \operatorname{B_{1}}{\left (x,y,z \right )}\right)}{\sqrt{\theta_{x}^{2} + \theta_{y}^{2} + 1}}  \nonumber\\
        &~~ + q\, v \left(- \theta_{y} \operatorname{B_{2}}{\left (x,y,z \right )} - \operatorname{B_{3}}{\left (x,y,z \right )}\right) \sqrt{\theta_{x}^{2} + \theta_{y}^{2} + 1} \nonumber\\
        \frac{\partial f_{\theta_y}}{\partial \theta_y} &=
        \frac{q\, \theta_{y}\, v \left(- \theta_{x} \left(\theta_{y} \operatorname{B_{2}}{\left (x,y,z \right )} + \operatorname{B_{3}}{\left (x,y,z \right )}\right) + \left(\theta_{y}^{2} + 1\right) \operatorname{B_{1}}{\left (x,y,z \right )}\right)}{\sqrt{\theta_{x}^{2} + \theta_{y}^{2} + 1}} \nonumber\\
        &~~ + q\, v \left(- \theta_{x} \operatorname{B_{2}}{\left (x,y,z \right )} + 2 \theta_{y} \operatorname{B_{1}}{\left (x,y,z \right )}\right) \sqrt{\theta_{x}^{2} + \theta_{y}^{2} + 1} \nonumber\\
        \frac{\partial f_{\theta_y}}{\partial q} &=
        v \left(- \theta_{x} \left(\theta_{y} \operatorname{B_{2}}{\left (x,y,z \right )} + \operatorname{B_{3}}{\left (x,y,z \right )}\right) + \left(\theta_{y}^{2} + 1\right) \operatorname{B_{1}}{\left (x,y,z \right )}\right) \sqrt{\theta_{x}^{2} + \theta_{y}^{2} + 1} \nonumber
    \end{align}
    \begin{equation*}
        \frac{\partial f_q}{\partial x} = 0,~
        \frac{\partial f_q}{\partial y} = 0,~
        \frac{\partial f_q}{\partial \theta_x} = 0,~
        \frac{\partial f_q}{\partial \theta_y} = 0,~
        \frac{\partial f_q}{\partial q} = 0\,.
    \end{equation*}

\newpage

It's worth noting that $\frac{\partial \textbf{B}}{\partial x}$, $\frac{\partial \textbf{B}}{\partial y}$ and $\frac{\partial \textbf{B}}{\partial z}$ are in practice very small.
Due to this, Equations \eqref{eq:tf_dtx_dx}, \eqref{eq:tf_dtx_dy}, \eqref{eq:tf_dty_dx} and \eqref{eq:tf_dty_dy} are redefined as:
    \begin{equation*}
        \frac{\partial f_{\theta_x}}{\partial x} = 0, ~
        \frac{\partial f_{\theta_x}}{\partial y} = 0, ~
        \frac{\partial f_{\theta_y}}{\partial x} = 0, ~
        \frac{\partial f_{\theta_y}}{\partial y} = 0\,.
    \end{equation*}
All of these equations of motions are then solved via Runge Kutta 4 as explained in Section \ref{ssec:framework_rk4}.

After transporting from one measurement plane to the next, EKF's \textbf{Update} phase must be applied.
The input state vector is again the same $\mathbf{x_{k|k-1}}$, with its related covariance matrix denoted as $P_{k|k-1}$, and the measurement vector $\mathbf{z}_k$, for each measurement plane $z$, is defined as:
    \begin{equation*}
        \mathbf{z}_k(z) = \begin{pmatrix}x_m\\ u\\ t\\ l\\ w\end{pmatrix}\,,
    \end{equation*}

A measurement results from an effective hit to a particular wire where a local tilted coordinate system is defined in which $y=0$ for convenience.
Each of the scalar components denote different attributes related to the measurement itself:
    \begin{itemize}
        \item $x_m$: \textbf{measured $x$ at $z$}.
        Note that since $y$ is fixed to $0$ and the measurement plane $z$ is known, only $x$ needs to be measured.
        \item $u$: \textbf{uncertainty of the measurement}.
        Converting the time to distance from the moment the measurement is taken at the wire, a certain statistical uncertainty is related to the measurement, with a minimum close to $300$ [ns].
        \item $t$: \textbf{measurement tilt}.
        Tilt used to define the tilted coordinate system which can be either $30\degree$ or $-30\degree$.
        \item $l$: \textbf{length of the wire from which the measurement is obtained}.
        As can be seen in Figure \ref{fig:dc_vertical_cut} from Section \ref{ssec:framework_dc}, each superlayer in each sector has an isosceles trapezoid shape when viewed transverse to the beam line at the target location.
        This means that wires can have different lengths depending on where on the trapezoid they're strung, so this needs to be considered when using the measurements.
        \item $w$: \textbf{the wire's maximum sag}.
        This variable denotes the maximum possible sag caused by the catenary the wire naturally follows due to gravity.
    \end{itemize}
then, vector $\mathbf{h}_k(z)$ is computed:
    \begin{equation} \label{eq:framework_tf_h}
        \mathbf{h}_k(z) = \begin{pmatrix}1\\ -\text{tan}(t) - \frac{4w}{l} \cdot (1 - \frac{2y}{l})\\ 0\\ 0\\ 0\end{pmatrix}\,,
    \end{equation}
which is a vector used to define the transformation necessary to project the state vector into the local coordinate system of the measurement.

A $H_k$ matrix can then be computed from $\mathbf{h}_k(z)$ to update the covariance matrix:
    \begin{equation*}
        H_k = \mathbf{h}_k \mathbf{h}_k^T\,,
    \end{equation*}
then, to calculate the filtered covariance matrix $P_{k|k}$:
    \begin{equation*}
        P_{k|k} = \left( P_{k|k-1}^{-1} + \frac{H_k}{|u|} \right)^{-1}\,,
    \end{equation*}

From this new covariance matrix, a vector $\mathbf{k}_k$ can be computed:
    \begin{equation*}
        \mathbf{k}_k = \frac{P_{k|k} \times \mathbf{h}_k}{|u|}\,,
    \end{equation*}
using $\mathbf{h}_k(z)$ from Equation \eqref{eq:framework_tf_h}.
From this vector $\mathbf{k}_k$, both the state vector $\mathbf{x}_{k|k}$ and the quadratic error $\chi^2_k$ (described in Appendix \ref{add:errors}) can be obtained:
    \begin{align*}
        \mathbf{x}_{k|k} &= \mathbf{x}_{k|k-1} + (x_m - (x - h_{k(2)})) \cdot \mathbf{k}_k\\
        \chi^2_k &= \chi^2_{k-1} + \frac{(x_m - (x - h_{k(2)}))^2}{|u|}\,,
    \end{align*}
where $h_{k(2)}$ is the second scalar component of the vector $\mathbf{h}_k(z)$ from Equation \eqref{eq:framework_tf_h}.
The quadratic error obtained can be used to evaluate how well the fit is going and, when the EKF is finished, to measure the quality of the fit.

The pseudocode for this particular implementation of the EKF for CLAS12 is presented:

\newpage

    \begin{algorithm}[H]
        \caption{Kalman Filter}
        \SetAlgoLined
        \DontPrintSemicolon
        \SetKwProg{Def}{def}{:}{}
        $Z \gets$ array of measurement planes\;
        $m \gets$ array of measurements\;
        $\mathbf{s}_0 \gets$ initial state vector\;
        $S_0 \gets$ initial covariance matrix\;
        $I \gets$ total number of iterations\;
    
        \Def{\upshape run\_fitter$(Z, m, \mathbf{s}, S, I)$}{
            $\chi_b^2 \gets\infty$\;
            $K \gets \text{size}(Z)-1$\;
            \For{$i\gets0$ \KwTo $I$}{
                $\chi_c^2 \gets0$\;
                \uIf{$i > 0$}{
                    $\mathbf{s}_0, S_0 \gets \text{transport}(K, 0, \mathbf{s}_K, S_K, \chi_c^2, Z)$\;
                }
                \For{$k\gets0$ \KwTo $K$}{
                    $\mathbf{s}_{k+1}, S_{k+1} \gets \text{transport}(k, k+1, \mathbf{s}_k, S_k, Z)$\;
                    $\mathbf{s}_{k+1}, S_{k+1}, \chi_c^2 \gets \text{filter}(\mathbf{s}_{k+1}, S_{k+1}, m_{k+1}, \chi_c^2)$\;
                }
                \uIf{\upshape $|\mathbf{s}_b.Q-\mathbf{s}_K.Q| < 5\cdot10^{-4}$ and 
                    $|\mathbf{s}_b.x-\mathbf{s}_K.x| < 10^{-4}$ and 
                    $|\mathbf{s}_b.y-\mathbf{s}_K.x| < 10^{-4}$ and 
                    $|\mathbf{s}_b.\theta_x-\mathbf{s}_K.\theta_x| < 10^{-6}$ and 
                    $|\mathbf{s}_b.\theta_y-\mathbf{s}_K.\theta_y| < 10^{-6}$}{
                    $i = I$\;
                }
                \uIf{$\chi_c^2 < \chi_b^2$}{
                    $\chi_b^2 \gets \chi_c^2$\;
                    $\mathbf{s}_b \gets \mathbf{s}_K$\;
                }
            }
            % $\chi^2 \gets \text{getFinalChi2()}$ % Note: Describing this is really hard and probably worth less than its benefit imo.
            \Return{$\mathbf{s}_K$, $S_K$, $\chi_b^2$}
        }
    \end{algorithm}
    
\newpage

    \begin{algorithm}[H]
        \caption{Transport}
        \SetAlgoLined
        \DontPrintSemicolon
        \SetKwProg{Def}{def}{:}{}
        $i \gets$ initial state ($k-1|k-1$)\;
        $f \gets$ final state ($k|k-1$)\;
        $\mathbf{s}_i \gets$ state vector at $i$\;
        $S_i \gets$ covariance matrix at $i$\;
        $\chi^2 \gets \chi^2$ error\;
        $Z \gets$ array of measurement planes\;
        \Def{\upshape transport$(i, f, \mathbf{s}_i, S_i, \chi^2, Z)$}{
            $\mathbf{s}_f \gets \mathbf{s}_i$\;
            $S_f \gets S_i$\;
            $h \gets 1$ \CommentSty{// step size}\;
            $z \gets Z_i$\;
            % $a \gets \text{sign}(Z_f - Z_i)$\;
            \While{\upshape $\text{sign}(Z_f - Z_i)\cdot \mathbf{s}_f.z < \text{sign}(Z_f - Z_i)\cdot Z_f$}{
                $h_{\text{signed}} \gets \text{sign}(Z_f - Z_i)\cdot h$\;
                \uIf{\upshape $\text{sign}(Z_f - Z_i)\cdot (\mathbf{s}_f.z + h_{\text{signed}}) > \text{sign}(Z_f - Z_i)\cdot Z_f$}{
                    $h_{\text{signed}} \gets \text{sign}(Z_f - Z_i)\cdot |Z_f - \mathbf{s}_f.z|$\;
                }
                $z, \mathbf{s}_f, S_f, \chi^2 \gets \text{rk4\_transport}(z, h_{\text{signed}}, \mathbf{s}_f, S_f, \chi^2)$\;
            }
            \Return{$\mathbf{s}_f$, $S_f$}
        }
    \end{algorithm}

    \begin{algorithm}[H]
        \caption{Filter}
        \SetAlgoLined
        \DontPrintSemicolon
        \SetKwProg{Def}{def}{:}{}
        $\mathbf{s}_i \gets$ state vector at $i$ ($k|k-1$)\;
        $S_i \gets$ covariance matrix at $i$ ($k|k-1$)\;
        $\mathbf{m}_k \gets$ measurement vector at $k$\;
        $\chi^2 \gets \chi^2$ error before running the filter\;
        \Def{\upshape filter$(\mathbf{s}_i, S_i, \mathbf{m}_k, \chi^2)$}{
            $\mathbf{h} \gets \text{get\_h}(\mathbf{s}_i.y, \mathbf{m}_k.t, \mathbf{m}_k.w, \mathbf{m}_k.l)$\;
            $S_f \gets S_i - \frac{S_i\mathbf{h}\mathbf{h}^TS_i}{1 + \mathbf{h}^TS_i\mathbf{h}}$\;
            $\mathbf{k} \gets <0, 0, 0, 0, 0>$\;
            \For{$j \gets 1$ \KwTo $5$}{
                $k_j \gets \frac{(h_0\cdot S_{f(j,0)} + h_1\cdot S_{f(j,1)})}{|\mathbf{m}_k.u|}$\;
            }
            $c \gets \text{get\_correction}(\mathbf{s}_i.x, \mathbf{s}_i.y, \mathbf{m}_k.t, \mathbf{m}_k.w, \mathbf{m}_k.l)$\;
            $\chi^2 \gets \chi^2 + \frac{(\mathbf{m}_k.x - c)^2}{|\mathbf{m}_k.u|}$\;
            $\mathbf{s}_f \gets \mathbf{s}_i + \mathbf{k}\cdot (\mathbf{m}_k.x - c)$\;
            \Return{$\mathbf{s}_f$, $S_f$, $\chi^2$}
        }
        \Def{\upshape get\_h$(y,t,w,l)$}{
            \Return{$<1, -\tan(6t) - w\cdot\frac{4}{l}\cdot\big(1-\frac{2y}{l}\big), 0, 0, 0>$}
        }
        \Def{\upshape get\_correction$(x,y,t,w,l)$}{
            \Return{$x - \tan(6t)\cdot y - w\cdot\big(1-\frac{2y}{l}\big)^2$}
        }
    \end{algorithm}