\secnumberlesssection{ABSTRACT}
\noindent \textbf{Abstract---}
Particle accelerators and detectors are one of the main sources of data for High Energy Physics studies and analysis in general, and thus are tools that require work and maintenance to keep running in optimal conditions.
In light of this, special care must be put into the software dedicated to the reconstruction of the detected events, and thus its optimization is essential to keep the production chain running efficiently. \\
This document presents this exact optimization, which is done by focusing on different elements associated with the components that take the most time in the offline reconstruction software of the CLAS12 detector. \\
The results obtained in the project are highly favorable, with a reduction of half the running time of the entire software.
These results are relevant because they allow for a much faster reconstruction of the available data, and thus accelerating the whole production chain.
A favorable part of the optimizations applied is that they can be useful for other institutions working with particle detectors.

\noindent \textbf{Keywords---} Particle detectors; Jefferson Laboratories; CLAS12; Drift chambers; Event reconstruction software.

\vspace{1.0cm}

\secnumberlesssection{RESUMEN}
\noindent \textbf{\emph{Resumen}---}
Los aceleradores y detectores de part\'iculas son una de las principales fuentes de data para el estudio y an\'alisis de la f\'isica de part\'iculas de alta energ\'ia, y por tanto son herramientas que requieren trabajo y mantenimiento para correr en condiciones \'optimas.
En vista de esto, especial atenci\'on se debe poner en el \textit{software} dedicado a la reconstrucci\'on de los eventos detectados, y por tanto su optimizaci\'on es esencial para mantener la l\'inea de producci\'on corriendo eficientemente.\\
Este documento presenta esta optimizaci\'on, la cual se realiza enfoc\'andose en los distintos elementos asociados a los componentes que m\'as tiempo toman en el \textit{software} de reconstrucci\'on \textit{offline} del detector CLAS12. \\
Los resultados obtenidos en el proyecto son altamente favorables, viendo una reducci\'on de la mitad del tiempo de la totalidad del \textit{software}.
Estos resultados son relevantes porque permiten una reconstrucci\'on mucho m\'as r\'apidos de los datos disponibles, y por tanto aceleran la l\'inea de producci\'on en su totalidad.
Una parte favorable de las optimizaciones aplicadas es que pueden ser \'utiles para otras instituciones que trabajan con detectores de part\'iculas.

\noindent \textbf{\emph{Palabras Clave}---} Detectores de part\'iculas; Jefferson Laboratories; CLAS12; \textit{Drift Chambers}; \textit{Software} de reconstrucci\'on de eventos.